 
\documentclass[10pt,twocolumn]{article}

    \usepackage{fullpage}          %This will give you the standard 1in. margins, 12pt font, single spaced, page numbers, ect.  
    \usepackage{url}                 %This is intended if you have a website in the bibliography
    \usepackage{graphicx}         %This is if you want to include a picture in your document

    \usepackage{amsmath, amsthm, amssymb}
    \usepackage{epsfig}
    \usepackage{pslatex}
	\usepackage[font=small,labelfont=bf]{caption}

    \title{$n$-Body Simulations of Newtonian Gravitation}
    \author{Thomas Dickerson}
    \begin{document}
    \maketitle

    \begin{abstract}
    This document contains a summary of the work performed in completion of the requirements for the Saint Michael's
	College Honors Program Senior Capstone Project in Physics. The topic for the project discussed herein concerns
	techniques for the numerical simulation of $n$-body systems governed by Newtonian gravitational interactions.
    \end{abstract}
     
    \section{Introduction}
	\label{sec:introduction}
	The goal of this project was to investigate techniques for the numerical simulation of $n$-body systems governed by
	Newtonian gravitational interactions and their associated tradeoffs in the computational complexity and numerical precision
	of such a simulation.
	
	While many special cases of this problem are well studied, with exact analytical solutions (for many $2$-body problems)
	or good analytical approximations in certain limiting cases  (e.g. Keplerian motion when the system is dominated by a large central mass),
	the goal of this project was to implement a general simulator for all $n$-body systems where Newtonian gravitational potentials provide a
	good approximation of General Relativity. This potential is given by
	\begin{equation}
	U(x_{1} \cdots x_{n}) = G \sum^{n}_{j=1} \sum^{j-1}_{i=1} \frac{m_{i} m_{j}}{\left|x_{i} - x_{j}\right|}.
	\end{equation}
	
	The works referenced in the bibliography provide a useful (though incomplete) starting point for developing the tools necessary
	to work with such problems. See ~\cite{AbiaSanzSerna,BarnesHut,Brizard,SODEII,McLachlanAtela} for more information. \S\ref{sec:motivation}
	discusses broadly the motivation for this investigation. \S\ref{sec:challenges} discusses the essential challenges to any such endeavour:
	\S\ref{sec:complexity} discusses techniques for dealing with those challenges caused by the inherent computational complexity of
	such problems, while \S\ref{sec:numerical} discusses techniques for dealing with those challenges whose source is in the inherent limitations
	of numerical methods. \S\ref{sec:conclusion}
	concludes this summary with a discussion of project outcomes.
	
	\section{Motivation}
	\label{sec:motivation}
	There are many possible motivating examples when dealing with $n$-body simulations in general, a significant number
	of which are related to gravitation and celestial dynamics. Examples include studies of the long-term stability of the Solar System,
	the eventual collision of the Milky Way and Andromeda galaxies, and a desire to make orbital predictions for Near-Earth Objects (NEOs)
	to make sure we're safe from catastrophic impacts. Moreover, virtually all $n$-body problems are analytically intractable, and must
	therefore be studied by numerical analysis and simulation. Newtonian gravitation provides a relatively simple test-bed for the development
	of such numerical methods, and many of the lessons and techniques developed in its study may be applied to more complex interaction
	potentials (for example, those in which relativistic effects must be accounted for).
	
	
	\section{Challenges}
	\label{sec:challenges}
	The challenges in constructing such a simulation can broadly be classified into one of two categories: those stemming from
	computational complexity, and those stemming from the inherent limitations of numerical methods. The computational complexity issue has a very straightforward source: the time required to simulate $n$ bodies is proportional to
	the number of interactions between them. In principle, this allows for up to $n \times (n-1)$ distinct interactions, but Newton's
	Third Law states that these must be pairwise symmetric between interacting bodies, reducing the total number of possible distinct
	interactions to $\frac{n \times (n-1)}{2}$. While smaller, this quantity still has a complexity which is $O(n^{2})$. Thus the time required
	to simulate a system grows much faster than the number of bodies being simulated.
	The limitations of numerical methods also have a straightforward source: physical laws and quantities are real-valued,
	and may require an infinite number of digits to express exactly, yet computing devices must operate under conditions of finite
	time and memory. Therefore, every computation must be truncated to some fixed number of digits, and rounding errors are thus likely to
	accumulate over time.
	
	\subsection{Dealing with Complexity}
	\label{sec:complexity}
	There are several approaches to dealing with the problems presented by the computational complexity of the problem. One answer is to
	not try and deal with it, and count on Moore's Law to double your processing power a year and a half down the road. While $O(2^{n})$
	growth certainly dominates $O(n^{2})$ growth, it is unclear how far into the future Moore's Law can continue, and even assuming it continues
	indefinitely, it will be a long time before such a strategy can be cost effective. It does have the advantage of requiring no analytical
	approximations about the problem (though the accuracy will still be bounded by numerical error).
	
	Most other approaches to dealing with the complexity issue rely on approximations where distant groups of bodies can be approximated
	as a single object when calculating their effect on another body. The Barnes-Hut algorithm does this by imposing a spatial hierarchy
	(for example a quad-tree or oct-tree data structure, depending on the dimensionality of the simulation), and requiring that leaf nodes contain
	at most $1$ body. Parent nodes that are suitably distant from a body under consideration are then treated as a weighted sum over their children
	(which is cached at each time step). This reduces the complexity to $O(n \times log(n))$ (where the base of the log is the number of children per node).
	The Fast-Multipole Method can reduce the complexity further to a very desirable $O(n)$, and relies on multipole expansions of certain equations
	related to the state of the system (i.e., Green's function, or the vector Helmholtz equation, depending on implementation and the interactions
	under consideration).
	
	\subsection{Dealing with Numerical Limitations}
	\label{sec:numerical}
	Numerical integration and differentiation in particular are limiting factors in the accuracy of individual steps in an $n$-body simulation.
	The straightforward approach is to use the conventional limit definitions of these operations (Eq. \ref{eq:limderiv} and Eq. \ref{eq:limint}), and instead of taking those limits
	to $\infty$ or $0$, respectively, take them to values which are finitely large or small, respectively.
	\begin{equation}
	\label{eq:limderiv}
	f'(x) = \lim_{h \to 0} \frac{f(a+h)-f(a)}{h}
	\end{equation}
	\begin{equation}
	\label{eq:limint}
	\int^{a}_{b} f(x) dx = \lim_{n \to \infty} \sum^{n}_{i=1} f(x_{i})\left(\frac{b-a}{n}\right)
	\end{equation}
	Doing so, yields the Riemann sum and (forward) finite difference approximations of these operations. Unfortunately, these na\"\i ve methods
	accumulate error linearly, in proportion to dt and to h. Slight improvements may be had by use of centered formulas, but errors will nonetheless
	accumulate quickly. Methods which retain higher order terms can be constructed to try and correct for the problem of undershooting
	(or overshooting) an estime, and, when well constructed, explicitly cancel lower-order errors. While such methods (e.g.
	the standard Runge-Kutta fourth-order method) have excellent accuracy, and so should do an excellent job of energy conservation,
	they tell us little about long-term preservation of qualitative features of the system. Energy conservation is well and good,
	but long term analysis is for naught without some surety that, for example, bounded orbits will stay bound. Thankfully a class of numerical
	methods known as symplectic integrators may be used for this purpose. Symplectic integrators may stretch or skew orbital features in phase space,
	but remain resistant to tearing of these features, even in the face of very long timesteps. However, it must always be kept in mind that
	higher quality numerical methods will require additional calculations of inter-body interactions (the primary expense in simulation), and so
	these too present a significant speed/accuracy tradeoff.
	
	\section{Results \& Conclusions}
	\label{sec:conclusion}
	The software developed for this project implements both the brute-force pairwise calculation of interactions, and the Barnes-Hut algorithm ~\cite{GravSim}.
	It also implements several different numerical methods for integration, and provides both numerical and analytic calculations of derivatives.
	It is currently incomplete, but the final version and some data for quantitative comparison of the relevant speed/accuracy tradeoffs should
	be available in time for the final presentation during exam week (May 7, 2013).


    \bibliography{summarybib}
    \bibliographystyle{plain}
\end{document}

